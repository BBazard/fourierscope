\documentclass{beamer}

\usepackage{pgfpages}
\usepackage[T1]{fontenc}
\usepackage[utf8]{inputenc}
\usepackage{lmodern}
\usepackage[francais]{babel}
\usepackage{graphicx}
\usepackage{booktabs}

\mode<presentation> {
\usetheme{Copenhagen}
\usecolortheme{beaver}

\setbeamertemplate{navigation symbols}{}
\expandafter\def\expandafter\insertshorttitle\expandafter{%
  \insertshorttitle\hfill%
  \insertframenumber\,/\,\inserttotalframenumber}

%\setbeameroption{show notes on second screen=right}
}

%%%%%%%%%%%%%%%%%%%%%%%%%%%%%%%%%%%%%%%%%%%%%%%

\title[Fourier Scope]{Fourier Scope}
\author{A Lescouet, B Bazard}
\institute[Télécom SudParis]{
  Télécom Sudparis\\
}
\date{\today}

%%%%%%%%%%%%%%%%%%%%%%%%%%%%%%%%%%%%%%%%%%%%%%%

\begin{document}

\begin{frame}
  \titlepage%
\end{frame}

\begin{frame}%[allowframebreaks]%
  \frametitle{Contents}
  \tableofcontents
\end{frame}

\section{Project overview}

\begin{frame}
  \frametitle{Dependencies}
  This project uses several libraries:
  \begin{itemize}
  \item \textbf{libtiff} for inputs and outputs in tiff format
  \item \textbf{libfftw3} for Fourier Transform computation
  \item \textbf{libgtest} for testing (not mandatory)
  \end{itemize}
\end{frame}

\begin{frame}
  \frametitle{The goal}
  This project aims to provide a efficient library to combine several
  high-quality images taken with a microscope into an ultra high quality
  image. %TODO: add an example with before and after
\end{frame}

\section{The algorithm}

\begin{frame}
  \frametitle{Overview}
  \begin{enumerate}
  \item Uses multiples images as input (tiff format)
  \item Computes the Fourier Transform of all theses
  \item Combines them in the frequency domain into a big one
  \item Computes the Inverse Fourier Transform to get the final results
  \item Output the image in tiff format
  \end{enumerate}
\end{frame}

%\begin{frame}
%  \frametitle{}
%\end{frame}

\section{Results}

\begin{frame}
  \frametitle{Input}
  \begin{center}
    \includegraphics[width=1\textwidth]{images/input}%
  \end{center}
\end{frame}

\begin{frame}
  \frametitle{Input v2}
  \begin{center}
    \includegraphics[width=1\textwidth]{images/input2}%
  \end{center}
\end{frame}

\begin{frame}
  \frametitle{Output}
  \begin{center}
    \includegraphics[width=0.65\textwidth]{images/output}%
  \end{center}
\end{frame}

\end{document}
